\documentclass[11pt]{scrartcl} % Font size
%%%%%%%%%%%%%%%%%%%%%%%%%%%%%%%%%%%%%%%%%
% Wenneker Assignment
% Structure Specification File
% Version 2.0 (12/1/2019)
%
% This template originates from:
% http://www.LaTeXTemplates.com
%
% Authors:
% Vel (vel@LaTeXTemplates.com)
% Frits Wenneker
%
% License:
% CC BY-NC-SA 3.0 (http://creativecommons.org/licenses/by-nc-sa/3.0/)
%
%%%%%%%%%%%%%%%%%%%%%%%%%%%%%%%%%%%%%%%%%

%----------------------------------------------------------------------------------------
%	PACKAGES AND OTHER DOCUMENT CONFIGURATIONS
%----------------------------------------------------------------------------------------

\usepackage{amsmath, amsfonts, amsthm} % Math packages
\usepackage{hyperref}
\usepackage{comment}
\usepackage{amsmath}
\usepackage{bbm}
\usepackage{algpseudocode, algorithm, algorithmicx}
\usepackage[toc,page]{appendix}
\usepackage[user,titleref]{zref}

\hypersetup{
   colorlinks=true,
   citecolor=red,
   linkcolor=cyan,
   filecolor=magenta,
   urlcolor=blue,
 }

\usepackage{url}
\usepackage{listings} % Code listings, with syntax highlighting

\usepackage[english]{babel} % English language hyphenation

\usepackage{graphicx} % Required for inserting images
\graphicspath{{Figures/}{./}} % Specifies where to look for included images (trailing slash required)

\usepackage{booktabs} % Required for better horizontal rules in tables

\numberwithin{equation}{section} % Number equations within sections (i.e. 1.1, 1.2, 2.1, 2.2 instead of 1, 2, 3, 4)
\numberwithin{figure}{section} % Number figures within sections (i.e. 1.1, 1.2, 2.1, 2.2 instead of 1, 2, 3, 4)
\numberwithin{table}{section} % Number tables within sections (i.e. 1.1, 1.2, 2.1, 2.2 instead of 1, 2, 3, 4)

\setlength\parindent{0pt} % Removes all indentation from paragraphs

\usepackage{enumitem} % Required for list customisation
\setlist{noitemsep} % No spacing between list items

%----------------------------------------------------------------------------------------
%	DOCUMENT MARGINS
%----------------------------------------------------------------------------------------

\usepackage{geometry} % Required for adjusting page dimensions and margins

\geometry{
	paper=a4paper, % Paper size, change to letterpaper for US letter size
	top=2.5cm, % Top margin
	bottom=3cm, % Bottom margin
	left=3cm, % Left margin
	right=3cm, % Right margin
	headheight=0.75cm, % Header height
	footskip=1.5cm, % Space from the bottom margin to the baseline of the footer
	headsep=0.75cm, % Space from the top margin to the baseline of the header
	%showframe, % Uncomment to show how the type block is set on the page
}

%----------------------------------------------------------------------------------------
%	FONTS
%----------------------------------------------------------------------------------------

\usepackage[utf8]{inputenc} % Required for inputting international characters
\usepackage[T1]{fontenc} % Use 8-bit encoding

% \usepackage{fourier} % Use the Adobe Utopia font for the document

%----------------------------------------------------------------------------------------
%	SECTION TITLES
%----------------------------------------------------------------------------------------

\usepackage{sectsty} % Allows customising section commands

\sectionfont{\vspace{6pt}\centering\normalfont\scshape} % \section{} styling
\subsectionfont{\normalfont\bfseries} % \subsection{} styling
\subsubsectionfont{\normalfont\bfseries} % \subsubsection{} styling
\paragraphfont{\normalfont\scshape} % \paragraph{} styling

%----------------------------------------------------------------------------------------
%	HEADERS AND FOOTERS
%----------------------------------------------------------------------------------------

\usepackage{scrlayer-scrpage} % Required for customising headers and footers

\ohead*{} % Right header
\ihead*{} % Left header
\chead*{} % Centre header

\ofoot*{} % Right footer
\ifoot*{} % Left footer
\cfoot*{\pagemark} % Centre footer

%-----------------------------------------------------------
%  Define new commands
%-----------------------------------------------------------
\DeclareMathOperator*{\argmax}{arg\,max}
\DeclareMathOperator*{\argmin}{arg\,min}
\DeclareMathOperator\supp{supp}

\newcommand*\Let[2]{\State #1 $\gets$ #2}

\DeclareMathOperator*{\E}{\mathbbm{E}}
\DeclareMathOperator*{\Z}{\mathbbm{Z}}
\DeclareMathOperator*{\R}{\mathbbm{R}} % Include the file specifying the document structure and custom commands

%----------------------------------------------------------------------------------------
%	TITLE SECTION
%----------------------------------------------------------------------------------------

\title{
	\normalfont\normalsize
	\textsc{Harvard Privacy Tools Project}\\ % Your university, school and/or department name(s)
	\vspace{25pt} % Whitespace
	\rule{\linewidth}{0.5pt}\\ % Thin top horizontal rule
	\vspace{20pt} % Whitespace
	{\huge Variance Sensitivity Proofs}\\ % The assignment title
	\vspace{12pt} % Whitespace
	\rule{\linewidth}{2pt}\\ % Thick bottom horizontal rule
	\vspace{12pt} % Whitespace
}

% \author{\LARGE} % Your name

\date{\normalsize\today} % Today's date (\today) or a custom date

\begin{document}

\maketitle

\begin{definition}
Let sample variance be defined as
$$ s^2 = \frac{1}{n-1} \sum_{i=1}^n (x_i - \mean)^2,$$
where $\mean$ refers to the sample mean of $x$.
\end{definition}

\section{Neighboring Definition: Change One}
\subsection{$\ell_1$-sensitivity}
\begin{lemma}
\label{lemma:meansum}
For arbitrary $a$,
$$ \sum_{i=1}^n (x_i - a)^2 = \sum_{i=1}^n (x_i - \bar{x})^2 + n(a-\bar{x})^2.$$
\end{lemma}

\begin{proof}
\begin{align*}
\sum_{i=1}^n (x_i - a)^2 &= \sum_{i=1}^n \left( (x_i - \bar{x}) - (a-\bar{x}) \right)^2\\
	&= \sum_{i=1}^n \left( (x_i - \bar{x})^2 -2(x_i - \bar{x})(a-\bar{x}) + (a-\bar{x})^2\right)\\
	&= \sum_{i=1}^n (x_i - \bar{x})^2 - 2\sum_{i=1}^n \left(x_ia-x_i\bar{x} -\bar{x}a + \bar{x}^2\right) + \sum_{i=1}^n \left( a^2 -2a\bar{x} + \bar{x}^2\right)\\
	&=  \sum_{i=1}^n (x_i - \bar{x})^2 -2a\sum_{i=1}^n x_i + 2\bar{x}\sum_{i=1}^n x_i + 2\bar{x}an - 2\bar{x}^2n + a^2n-2a\bar{x}n+\bar{x}^2n\\
	&=  \sum_{i=1}^n (x_i - \bar{x})^2 + a^2n-2a\bar{x}n+\bar{x}^2n\\
	&=  \sum_{i=1}^n (x_i - \bar{x})^2 + n(a-\bar{x})^2
\end{align*}

\end{proof}

\begin{theorem}
Let
$$ f(\x) = \sum_{i=1}^n (x_i - \bar{x})^2.$$
Then for $\x$ bounded between $m$ and $M,$ $f$ has sensitivity bounded above by
$$\frac{n-1}{n} (M-m)^2.$$
\end{theorem}

\begin{proof}
Consider databases $\xprime$ and $\xprimeprime$ which differ in a single point. For notational ease, call $\x$ the part of $\xprime$ and $\xprimeprime$
that is the same, and say that $\x$ contains $n$ points. WLOG say that the last data point in the database is the one that differs. I.e.,
$\xprime = \x \cup \{x_{n+1}\},$ and $\xprimeprime = \x \cup \{x'_{n+1}\}$. This proof assumes that a ``neighboring database" is one that differs in a single
data-point, so we will ultimately be comparing $f(\xprime)$ and $f(\xprimeprime)$. However, it is useful to first write $f(\xprime)$ in terms of $f(\x)$.
Note that

\begin{align}
\label{eqn:meanprime}
\bar{x}' &= \frac{1}{n+1} \sum_{i=1}^{n+1} x_i \nonumber \\
	&= \frac{n\bar{x} + x_{n+1}}{n+1}.
\end{align}

Then,
\begin{align}
\label{eq:varAddOnePoint}
f(\xprime) &= \sum_{i=1}^n (x_i - \bar{x}')^2 + (x_{n+1} - \bar{x}')^2 \notag\\
	&= \sum_{i=1}^n (x_i - \bar{x})^2 + n(\bar{x}'-\bar{x})^2 + (x_{n+1} - \bar{x}')^2 &&\text{(By Lemma \ref{lemma:meansum})} \notag\\
	&= f(\x) + n\left( \frac{n\bar{x} + x_{n+1}}{n+1}-\bar{x}\right)^2 + \left(x_{n+1} - \frac{n\bar{x} + x_{n+1}}{n+1} \right)^2 &&\text{(By Equation \ref{eqn:meanprime})} \notag\\
	&= f(\x) + n\left(\frac{x_{n+1}-\bar{x}}{n+1}\right)^2 + \left( \frac{nx_{n+1}-n\bar{x}}{n+1}\right)^2 \notag\\
	&= f(\x) + (x_{n+1}-\bar{x})^2 \frac{n+n^2}{(n+1)^2} \notag\\
	&= f(\x) + (x_{n+1}-\bar{x})^2 \frac{n}{n+1} \notag\\
\end{align}

Now, to bound the sensitivity of $f$, note that

\begin{align}
	\label{eq:l1_np1_bound}
\left\vert f(\xprime) - f(\xprimeprime) \right\vert &= \left\vert (x_{n+1}-\bar{x})^2 \frac{n}{n+1} - (x_{n+1}'-\bar{x})^2 \frac{n}{n+1} \right\vert \\
	&\le (M-m)^2 \frac{n}{n+1}. \nonumber
\end{align}
The bound in the final line follows from the case where $x_{n+1} = M$ (resp. $m$) and $\bar{x} = x'_{n+1} = m$ (resp. $M$). \newline

So we have a bound on the sensitivity of $f$ for a data set of size $n+1$, but traditionally we consider sensitivities
on a data set of size $n$.
Redefining $n+1$ as $n$ in the above equation gives
$$ (M-m)^2 \frac{n-1}{n}. $$
\end{proof}

\begin{corollary}
Sample variance has sensitivity bounded above by
$$ \frac{(M-m)^2 }{n}. $$
\end{corollary}

\subsection{$\ell_2$-sensitivity}
\begin{theorem}
	Let $X$ be a data set with $n$ elements, $x_1, \hdots, x_n$ and
	\[ f(X) = \sum_{i=1}^n (x_i - \bar{x})^2 \]
	be the sample variance.
	For $X$ bounded between $m$ and $M,$ $f$ has a global sensitivity of
	\[ \left( \frac{n-1}{n} (M-m)^2 \right)^2 \]
\end{theorem}

\begin{proof}
	We can pick up from statement~\ref{eq:l1_np1_bound}, switching from $\ell_1$ to $\ell_2$ norm and interpreting
	the data sets in question to be of size $n$ rather than $n+1$.
	\begin{align*}
		(f(x') - f(x''))^2
			&= \left( (x_{n}-\bar{x})^2 \frac{n-1}{n} - (x_{n}'-\bar{x})^2 \frac{n-1}{n} \right)^2 \\
			&= \left( \frac{n-1}{n} \right)^2 \left( (x_n - \bar{x})^2 - (x'_n - \bar{x})^2 \right)^2 \\
			&\leq \left( \frac{n-1}{n} \right)^2 \left( (M-m)^2 \right)^2 \\
			&= \left( \frac{n-1}{n} \right)^2 (M - m)^4 \\
			&= \left( \frac{n-1}{n} (M - m)^2 \right)^2.
	\end{align*}
\end{proof}

\section{Neighboring Definition: Add/Drop One}
\subsection{$\ell_1$-sensitivity}
 \begin{theorem}
 \label{thm:l1addsub}
 	Let $X$ be a data set with $n$ elements, $x_1, \hdots, x_n$ and
 	\[ f(X) = \sum_{i=1}^n (x_i - \bar{X})^2. \]
 	For $X$ bounded between $m$ and $M,$ $f$ has a global sensitivity of
 	\[ \frac{n}{n+1} (M-m)^2 \]
 \end{theorem}

 \begin{proof}
 	We must consider both adding and removing an element from $X$. \newline

 	Adding an element: \newline
 	Let $X' = X \cup x'_{n+1}$. Recall from Eq.~\ref{eq:varAddOnePoint} that for

	$$ f(x) = \sum_{i=1}^n (x_i - \bar{x})^2, $$
	$$ f(x') = f(x) + (x_{n+1} - \bar{x})^2 \frac{n}{n+1}.$$

	So,

	\begin{align}
	\label{eq:L1add}
	\left\vert f(x') - f(x) \right\vert &= \left\vert f(x) + (x_{n+1} - \bar{x})^2 \frac{n}{n+1} - f(x) \right\vert \nonumber\\
		&= \left\vert (x_{n+1} - \bar{x})^2 \frac{n}{n+1} \right\vert \nonumber \\
		&\le \left( M-m \right)^2 \frac{n}{n+1} 
	\end{align}

	Removing an element: \\
	Let $X' = X \setminus \{x_n\}$. Then, rewriting Eq.~\ref{eq:varAddOnePoint} with $n$ set to $n+1$ since ``$x$'' in this case is the greater set,
	$$ f(x) = f(x') + (x_n - \bar{x}')^2 \frac{n-1}{n}.$$

	Then,
	\begin{align}
	\label{eq:L1sub}
	\left\vert f(x) - f(x') \right\vert &= \left\vert f(x') + (x_n - \bar{x}')^2 \frac{n-1}{n} - f(x') \right\vert \nonumber\\
		&\le (M-m)^2 \frac{n-1}{n},\\
	\end{align}

	Note that for any $n \ge 1$,
	\begin{equation}
	\label{ineq}
	 \frac{n}{n + 1} > \frac{n-1}{n}.
	\end{equation}

	So, the worst-case bound always occurs in the ``add-one'' case, and the $\ell_1$-sensitivity of $f(\cdot)$ is in general bounded by

\begin{equation}
\label{eq:fBoundaddsub}
\left( M-m \right)^2 \frac{n}{n + 1}.
\end{equation}

 \end{proof}

 \begin{corollary}
 \label{cor:l1addsub}
	Sample variance has $\ell_1$ sensitivity bounded above by
	$$\left( M-m \right)^2 \frac{n}{n^2 - 1}. $$
 \end{corollary}
 
 \begin{proof}
 Define $f$ as in the statement of Theorem \ref{thm:l1addsub}. Note that sample variance is equal to 
 $$ \frac{f(x)}{n-1} $$. Then, from the bound in Equation \ref{eq:fBoundaddsub} it follows that the sensitivity of the sample variance is bounded from above by 
 $$ \left( M-m \right)^2 \frac{n}{(n-1)(n + 1)} = \left( M-m \right)^2 \frac{n}{n^2 - 1}. $$
 \end{proof}

\subsection{$\ell_2$-sensitivity}

\begin{theorem}
Sample variance has $\ell_2$ sensitivity bounded above by
	$$ \left( \left( M-m \right)^2 \frac{n}{n^2 - 1} \right)^2.$$
\end{theorem}
\begin{proof}
From the bounds in Equations \ref{eq:L1add} and \ref{eq:L1sub} and the inequality for $n \ge 1$ in Equation \ref{ineq}, because the bounds in Equations \ref{eq:L1add} and \ref{eq:L1sub} are positive it follows that the $\ell_2$ sensitivity of the variance is bounded by 
$$
\left( \left( M-m \right)^2 \frac{n}{n^2 - 1} \right)^2,
$$
where the change in constant from Equations \ref{eq:L1add} to $n/(n^2 - 1)$ comes from the $1/(n-1)$ in the definition of sample variance, as in the proof of Corollary \ref{cor:l1addsub}.
\end{proof}

\bibliographystyle{alpha}
\bibliography{mean}

\end{document}