\documentclass[11pt]{scrartcl} % Font size
%%%%%%%%%%%%%%%%%%%%%%%%%%%%%%%%%%%%%%%%%
% Wenneker Assignment
% Structure Specification File
% Version 2.0 (12/1/2019)
%
% This template originates from:
% http://www.LaTeXTemplates.com
%
% Authors:
% Vel (vel@LaTeXTemplates.com)
% Frits Wenneker
%
% License:
% CC BY-NC-SA 3.0 (http://creativecommons.org/licenses/by-nc-sa/3.0/)
%
%%%%%%%%%%%%%%%%%%%%%%%%%%%%%%%%%%%%%%%%%

%----------------------------------------------------------------------------------------
%	PACKAGES AND OTHER DOCUMENT CONFIGURATIONS
%----------------------------------------------------------------------------------------

\usepackage{amsmath, amsfonts, amsthm} % Math packages
\usepackage{hyperref}
\usepackage{comment}
\usepackage{amsmath}
\usepackage{bbm}
\usepackage{algpseudocode, algorithm, algorithmicx}
\usepackage[toc,page]{appendix}
\usepackage[user,titleref]{zref}

\hypersetup{
   colorlinks=true,
   citecolor=red,
   linkcolor=cyan,
   filecolor=magenta,
   urlcolor=blue,
 }

\usepackage{url}
\usepackage{listings} % Code listings, with syntax highlighting

\usepackage[english]{babel} % English language hyphenation

\usepackage{graphicx} % Required for inserting images
\graphicspath{{Figures/}{./}} % Specifies where to look for included images (trailing slash required)

\usepackage{booktabs} % Required for better horizontal rules in tables

\numberwithin{equation}{section} % Number equations within sections (i.e. 1.1, 1.2, 2.1, 2.2 instead of 1, 2, 3, 4)
\numberwithin{figure}{section} % Number figures within sections (i.e. 1.1, 1.2, 2.1, 2.2 instead of 1, 2, 3, 4)
\numberwithin{table}{section} % Number tables within sections (i.e. 1.1, 1.2, 2.1, 2.2 instead of 1, 2, 3, 4)

\setlength\parindent{0pt} % Removes all indentation from paragraphs

\usepackage{enumitem} % Required for list customisation
\setlist{noitemsep} % No spacing between list items

%----------------------------------------------------------------------------------------
%	DOCUMENT MARGINS
%----------------------------------------------------------------------------------------

\usepackage{geometry} % Required for adjusting page dimensions and margins

\geometry{
	paper=a4paper, % Paper size, change to letterpaper for US letter size
	top=2.5cm, % Top margin
	bottom=3cm, % Bottom margin
	left=3cm, % Left margin
	right=3cm, % Right margin
	headheight=0.75cm, % Header height
	footskip=1.5cm, % Space from the bottom margin to the baseline of the footer
	headsep=0.75cm, % Space from the top margin to the baseline of the header
	%showframe, % Uncomment to show how the type block is set on the page
}

%----------------------------------------------------------------------------------------
%	FONTS
%----------------------------------------------------------------------------------------

\usepackage[utf8]{inputenc} % Required for inputting international characters
\usepackage[T1]{fontenc} % Use 8-bit encoding

% \usepackage{fourier} % Use the Adobe Utopia font for the document

%----------------------------------------------------------------------------------------
%	SECTION TITLES
%----------------------------------------------------------------------------------------

\usepackage{sectsty} % Allows customising section commands

\sectionfont{\vspace{6pt}\centering\normalfont\scshape} % \section{} styling
\subsectionfont{\normalfont\bfseries} % \subsection{} styling
\subsubsectionfont{\normalfont\bfseries} % \subsubsection{} styling
\paragraphfont{\normalfont\scshape} % \paragraph{} styling

%----------------------------------------------------------------------------------------
%	HEADERS AND FOOTERS
%----------------------------------------------------------------------------------------

\usepackage{scrlayer-scrpage} % Required for customising headers and footers

\ohead*{} % Right header
\ihead*{} % Left header
\chead*{} % Centre header

\ofoot*{} % Right footer
\ifoot*{} % Left footer
\cfoot*{\pagemark} % Centre footer

%-----------------------------------------------------------
%  Define new commands
%-----------------------------------------------------------
\DeclareMathOperator*{\argmax}{arg\,max}
\DeclareMathOperator*{\argmin}{arg\,min}
\DeclareMathOperator\supp{supp}

\newcommand*\Let[2]{\State #1 $\gets$ #2}

\DeclareMathOperator*{\E}{\mathbbm{E}}
\DeclareMathOperator*{\Z}{\mathbbm{Z}}
\DeclareMathOperator*{\R}{\mathbbm{R}} % Include the file specifying the document structure and custom commands

%----------------------------------------------------------------------------------------
%	TITLE SECTION
%----------------------------------------------------------------------------------------

\title{
	\normalfont\normalsize
	\textsc{Harvard Privacy Tools Project}\\ % Your university, school and/or department name(s)
	\vspace{25pt} % Whitespace
	\rule{\linewidth}{0.5pt}\\ % Thin top horizontal rule
	\vspace{20pt} % Whitespace
	{\huge Count Sensitivity Proofs}\\ % The assignment title
	\vspace{12pt} % Whitespace
	\rule{\linewidth}{2pt}\\ % Thick bottom horizontal rule
	\vspace{12pt} % Whitespace
}

% \author{\LARGE} % Your name

\date{\normalsize\today} % Today's date (\today) or a custom date

\begin{document}

\maketitle

\begin{definition}
Let $\mathcal{X}$ be the universe of possible rows (individuals) and let $I: \mathcal{X} \rightarrow \{0,1\}$ be a predicate on rows. Let $x \in \mathcal{X}^n$ be a dataset. Then a count over $x$ is defined as 
$$ q(x) = \sum_{i=1}^n I(x_i).$$
\end{definition}

\begin{definition}
Let $q_1, \ldots, q_k$ be a series of counts with predicates $I_1, \ldots, I_k$. These counts are disjoint for every row in the database, only one of them evaluates to 1. In other words, they are disjoint if $\forall x_i \in \mathcal{X},$ 
$$ \sum_{j=1}^k I_j(x_i) = 1.$$
\end{definition}

\section{Neighboring Definition: Change One}

% l1 sensitivity
\subsection{$\ell_1$-sensitivity}

\begin{theorem}
\label{thm:change1L1}
A single count query has sensitivity 1. A series of $k$ disjoint counts has sensitivity 2.
\end{theorem}

\begin{proof}
Let $q$ be a count query with predicate $I$, and let $x'$ be equal to $x$ with point $x_i$ changed to $x_i'$. Then,

\begin{align*}
\left\vert q(x') - q(x) \right\vert &= \left\vert \sum_{j=1}^n I(x_j') - \sum_{j=1}^n I(x_j) \right\vert \\
	&= \left\vert \sum_{j \ne i}^n I(x_j) + I(x_i') - \sum_{j \ne i}^n I(x_j) - I(x_i) \right\vert \\
	&= \left\vert I(x_i') - I(x_i) \right\vert \\
	& \le 1.
\end{align*}

Consider a series of $k$ disjoint count queries $\mathbf{q} = \{q_1, \ldots, q_k\}$ on the same databases $x$ and $x'$. Note that since the counts are disjoint, only one query will be affected by point $x_i$ and one will be affected by point $x_i'$. Call these affected queries $q_i$ and $q_j$ respectively. Then,

\begin{align*}
\left\vert \mathbf{q}(x) - \mathbf{q}(x') \right\vert &= \left\vert \left(q_1(x) - q_1(x')\right) + \ldots + \left(q_k(x) - q_k(x')\right) \right\vert \\
	&= \left\vert \right(q_i(x) - q_i(x')\left) + \right(q_i(x) - q_j(x')\left) \right\vert \\
	&\le 2
\end{align*}
\end{proof}

% l2 sensitivity
\subsection{$\ell_2$-sensitivity}

\begin{theorem}
A single count query has sensitivity 1. A series of $k$ disjoint counts has sensitivity 2.
\end{theorem}

\begin{proof}
From the proof of Theorem \ref{thm:change1L1}, the difference between counts on two neighboring databases is at most 1. Squaring this gives the same value. For a series of $k$ disjoint counts,
\begin{align*}
\left\vert \mathbf{q}(x) - \mathbf{q}(x') \right\vert_2 &= \left\vert \left(q_1(x) - q_1(x')\right)^2 + \ldots + \left(q_k(x) - q_k(x')\right)^2 \right\vert \\
 & \le \left\vert 1^2 + 1^2 \right\vert \\
 &= 2
 \end{align*}
\end{proof}

\section{Neighboring Definition: Add/Drop One}
% l1 sensitivity
\subsection{$\ell_1$-sensitivity}
\begin{theorem}
\label{thm:addDropL1}
A single count query has sensitivity 1. A series of $k$ disjoint counts also has sensitivity 1.
\end{theorem}

\begin{proof}
Let $q$ be a count query with predicate $I$, and let $x'$ be equal to database $x$ with point $x_i$ removed. Then 
\begin{align*}
\left \vert q(x) - q(x') \right\vert &= \left\vert \sum_{j=1}^n I(x_j) - \sum_{j\ne i}^n I(x_j) \right\vert\\
	&= \left\vert \sum_{j \ne i} I(x_j) + I(x_i) - \sum_{j\ne i} I(x_j) \right\vert\\
	&\le 1.
\end{align*}

A nearly identical argument holds for adding a point. 
Consider a series of $k$ disjoint count queries $\mathbf{q} = \{q_1, \ldots, q_k\}$ and consider database $x'$ equal to database $x$ with point $x_i$ removed. Note that only a single one of the $k$ queries will be affected by the change from $x$ to $x'$, so
\begin{align*}
\left\vert \mathbf{q}(x) - \mathbf{q}(x') \right\vert &= \left\vert \left(q_1(x) - q_1(x')\right) + \ldots + \left(q_k(x) - q_k(x')\right) \right\vert \\
	&\le 1.
\end{align*}
The same argument holds for $x'$ equal to $x$ with a single point added.
\end{proof}
% l2 sensitivity
\subsection{$\ell_2$-sensitivity}
\begin{theorem}
A single count query has sensitivity 1. A series of $k$ disjoint counts also has sensitivity 1.
\end{theorem}

\begin{proof}
Squaring the sensitivity bounds from Theorem \ref{thm:addDropL1} gives 1 as an upper bound on the $\ell_2$ sensitivity for both a single count and a series of $k$ disjoint counts.
\end{proof}
% \bibliographystyle{alpha}
% \bibliography{mean}

\end{document}