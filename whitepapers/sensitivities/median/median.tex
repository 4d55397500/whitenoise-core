\documentclass[11pt]{scrartcl} % Font size
%%%%%%%%%%%%%%%%%%%%%%%%%%%%%%%%%%%%%%%%%
% Wenneker Assignment
% Structure Specification File
% Version 2.0 (12/1/2019)
%
% This template originates from:
% http://www.LaTeXTemplates.com
%
% Authors:
% Vel (vel@LaTeXTemplates.com)
% Frits Wenneker
%
% License:
% CC BY-NC-SA 3.0 (http://creativecommons.org/licenses/by-nc-sa/3.0/)
%
%%%%%%%%%%%%%%%%%%%%%%%%%%%%%%%%%%%%%%%%%

%----------------------------------------------------------------------------------------
%	PACKAGES AND OTHER DOCUMENT CONFIGURATIONS
%----------------------------------------------------------------------------------------

\usepackage{amsmath, amsfonts, amsthm} % Math packages
\usepackage{hyperref}
\usepackage{comment}
\usepackage{amsmath}
\usepackage{bbm}
\usepackage{algpseudocode, algorithm, algorithmicx}
\usepackage[toc,page]{appendix}
\usepackage[user,titleref]{zref}

\hypersetup{
   colorlinks=true,
   citecolor=red,
   linkcolor=cyan,
   filecolor=magenta,
   urlcolor=blue,
 }

\usepackage{url}
\usepackage{listings} % Code listings, with syntax highlighting

\usepackage[english]{babel} % English language hyphenation

\usepackage{graphicx} % Required for inserting images
\graphicspath{{Figures/}{./}} % Specifies where to look for included images (trailing slash required)

\usepackage{booktabs} % Required for better horizontal rules in tables

\numberwithin{equation}{section} % Number equations within sections (i.e. 1.1, 1.2, 2.1, 2.2 instead of 1, 2, 3, 4)
\numberwithin{figure}{section} % Number figures within sections (i.e. 1.1, 1.2, 2.1, 2.2 instead of 1, 2, 3, 4)
\numberwithin{table}{section} % Number tables within sections (i.e. 1.1, 1.2, 2.1, 2.2 instead of 1, 2, 3, 4)

\setlength\parindent{0pt} % Removes all indentation from paragraphs

\usepackage{enumitem} % Required for list customisation
\setlist{noitemsep} % No spacing between list items

%----------------------------------------------------------------------------------------
%	DOCUMENT MARGINS
%----------------------------------------------------------------------------------------

\usepackage{geometry} % Required for adjusting page dimensions and margins

\geometry{
	paper=a4paper, % Paper size, change to letterpaper for US letter size
	top=2.5cm, % Top margin
	bottom=3cm, % Bottom margin
	left=3cm, % Left margin
	right=3cm, % Right margin
	headheight=0.75cm, % Header height
	footskip=1.5cm, % Space from the bottom margin to the baseline of the footer
	headsep=0.75cm, % Space from the top margin to the baseline of the header
	%showframe, % Uncomment to show how the type block is set on the page
}

%----------------------------------------------------------------------------------------
%	FONTS
%----------------------------------------------------------------------------------------

\usepackage[utf8]{inputenc} % Required for inputting international characters
\usepackage[T1]{fontenc} % Use 8-bit encoding

% \usepackage{fourier} % Use the Adobe Utopia font for the document

%----------------------------------------------------------------------------------------
%	SECTION TITLES
%----------------------------------------------------------------------------------------

\usepackage{sectsty} % Allows customising section commands

\sectionfont{\vspace{6pt}\centering\normalfont\scshape} % \section{} styling
\subsectionfont{\normalfont\bfseries} % \subsection{} styling
\subsubsectionfont{\normalfont\bfseries} % \subsubsection{} styling
\paragraphfont{\normalfont\scshape} % \paragraph{} styling

%----------------------------------------------------------------------------------------
%	HEADERS AND FOOTERS
%----------------------------------------------------------------------------------------

\usepackage{scrlayer-scrpage} % Required for customising headers and footers

\ohead*{} % Right header
\ihead*{} % Left header
\chead*{} % Centre header

\ofoot*{} % Right footer
\ifoot*{} % Left footer
\cfoot*{\pagemark} % Centre footer

%-----------------------------------------------------------
%  Define new commands
%-----------------------------------------------------------
\DeclareMathOperator*{\argmax}{arg\,max}
\DeclareMathOperator*{\argmin}{arg\,min}
\DeclareMathOperator\supp{supp}

\newcommand*\Let[2]{\State #1 $\gets$ #2}

\DeclareMathOperator*{\E}{\mathbbm{E}}
\DeclareMathOperator*{\Z}{\mathbbm{Z}}
\DeclareMathOperator*{\R}{\mathbbm{R}} % Include the file specifying the document structure and custom commands

%----------------------------------------------------------------------------------------
%	TITLE SECTION
%----------------------------------------------------------------------------------------

\title{
	\normalfont\normalsize
	\textsc{Harvard Privacy Tools Project}\\ % Your university, school and/or department name(s)
	\vspace{25pt} % Whitespace
	\rule{\linewidth}{0.5pt}\\ % Thin top horizontal rule
	\vspace{20pt} % Whitespace
	{\huge Median Sensitivity Proofs}\\ % The assignment title
	\vspace{12pt} % Whitespace
	\rule{\linewidth}{2pt}\\ % Thick bottom horizontal rule
	\vspace{12pt} % Whitespace
}

% \author{\LARGE} % Your name

\date{\normalsize\today} % Today's date (\today) or a custom date

\begin{document}

\maketitle

\begin{definition}
The sample median of a database $X = (x_1, \hdots, x_n)$ is given by
\[ f(X) = \frac{\tilde{x}_l + \tilde{x}_u}{2} \]
where $l = \lfloor \frac{n+1}{2} \rfloor$ and $u = \lceil \frac{n+1}{2} \rceil$ and
$\tilde{X} = (\tilde{x}_1, \hdots, \tilde{x}_n)$ is the sorted version of $X$.
\end{definition}

\section{Neighboring Definition: Change One}

% l1 sensitivity
\subsection{$\ell_1$-sensitivity}
\begin{theorem}
	Let the database $X$ have elements bounded above by $M$ and below by $m$.
	Then the $\ell_1$-global sensitivity in the change-one model of the median is
	\[
		\Delta f(\cdot) =
			\begin{cases}
				\frac{M - m}{2}, \text{ if } n \equiv 0 \pmod{2} \\
				M-m \text{ if } n \equiv 1 \pmod{2}
			\end{cases}
	\]
\end{theorem}

\begin{proof}
	First consider the case where $n$ is such that $n \equiv 0 \pmod{2}$. Then our worst-case scenario is that exactly half of the data elements in $X$ are $m$ and the other half are $M$. In this case,
	$f(X) = \frac{m + M}{2}$. \newline

	Now consider an $X'$ which is identical to $X$ but with one additional element with value $M$.\footnote{The result holds if it is switched from $M$ to $m$.}
	Then we have that $f(X') = M$. Because we are looking at our worst-case pairing of $X,X'$,
	we know that
	\[ \Delta f(\cdot) = \max_{X,X'} |f(X') - f(X)| = \big\vert M - \frac{m+M}{2} \big\vert = \frac{M-m}{2}. \]

	Now consider the case where $n$ is such that $n \equiv 1 \pmod{2}$.
	Then our worst-case scenario is that $\frac{n-1}{2}$ of the elements of $X$ are $m$,
	$\frac{n-1}{2}$ of the elements of $X$ are $M$, and the remaining element of $X$ is $m$ (WLOG).
	In this setting, $f(X) = m$.
	Let $X'$ be identical to $x$ but with one element switched from $m$ to $M$.
	Then $f(X') = M$ and we have that
	\[ \Delta f(\cdot) = \max_{X,X'} |f(X') - f(X)| = |M - m|. \]
\end{proof}

% l2 sensitivity
\subsection{$\ell_2$-sensitivity}
\begin{theorem}
	Let the database $X$ have elements bounded above by $M$ and below by $m$.
	Then the global $\ell_2$-sensitivity in the change-one model of the median is
	\[
		\Delta f(\cdot) =
			\begin{cases}
				\left( \frac{M - m}{2} \right)^2, \text{ if } n \equiv 0 \pmod{2} \\
				(M-m)^2 \text{ if } n \equiv 1 \pmod{2}
			\end{cases}
	\]
\end{theorem}

\begin{proof}
	The logic follows exactly from the proof of the $\ell_1$ sensitivity,
	just with the norm in the last line of each statement switched from 1 to 2.
\end{proof}

\section{Neighboring Definition: Add/Drop One}

% l1 sensitivity
\subsection{$\ell_1$-sensitivity}

\begin{theorem}
Let the database $X$ have elements bounded above by $M$ and below by $m$. Then the global $\ell_1$-sensitivity in the add/drop-one model of the median is
$$ \Delta f(\cdot) = \frac{M-m}{2}.$$
\end{theorem}

\begin{proof}

First consider the case where $n$ is such that $n \equiv 0 \pmod{2}$. Then our worst-case scenario is that exactly half of the data elements in $X$ are $m$ and the other half are $M$. In this case, $f(X) = \frac{m + M}{2}$. \\

Now consider an $X'$ which is identical to $X$ but with a single additional element $M$.Then $f(X') = M$. Note that if instead $X'$ was identical to $X$ with a single element \textit{removed}, the worst case is that a single $m$ is removed and again $f(X') = M$.\footnote{The results holds if the point added/removed is switched from $M$ to $m$.}  In either case,
$$ \left\vert f(X') - f(X) \right\vert = \left\vert M - \frac{M-m}{2} \right\vert = \frac{M-m}{2}.$$

Now consider the case where $n$ is such that $n \equiv 1 \pmod{2}$.
Then our worst-case scenario is that $\frac{n-1}{2}$ of the elements of $X$ are $m$, and $\frac{n+1}{2}$ of the elements of $X$ are $M$ (WLOG). In this setting, $f(X) = M$. Let $X'$ be identical to $x$ but with one element $m$ added. Then $f(X') = \frac{M-m}{2}$. Note that if instead $X'$ was identical to $X$ with a single element removed, the worst case is that a single $M$ is removed and again $f(X') = \frac{M-m}{2}$ (WLOG). In either case, 

$$ \left\vert f(X) - f(X') \right\vert = \left\vert M - \frac{M-m}{2} \right\vert = \frac{M-m}{2}.$$ 

So, in general, 

$$ \Delta f(\cdot) = \max_{X,X'} \left\vert f(X') - f(X) \right\vert = \frac{M-m}{2}. $$
\end{proof}

% l2 sensitivity
\subsection{$\ell_2$-sensitivity}

\begin{theorem}
Let the database $X$ have elements bounded above by $M$ and below by $m$. Then the global $\ell_2$-sensitivity in the add/drop-one model of the median is
$$ \Delta f(\cdot) = \left(\frac{M-m}{2}\right)^2.$$
\end{theorem}

\begin{proof}
The logic follows exactly from the proof of the $\ell_1$ sensitivity, just with the norm switched from 1 to 2.

\end{proof}
\end{document}