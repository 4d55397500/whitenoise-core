\documentclass[11pt]{scrartcl} % Font size
%%%%%%%%%%%%%%%%%%%%%%%%%%%%%%%%%%%%%%%%%
% Wenneker Assignment
% Structure Specification File
% Version 2.0 (12/1/2019)
%
% This template originates from:
% http://www.LaTeXTemplates.com
%
% Authors:
% Vel (vel@LaTeXTemplates.com)
% Frits Wenneker
%
% License:
% CC BY-NC-SA 3.0 (http://creativecommons.org/licenses/by-nc-sa/3.0/)
%
%%%%%%%%%%%%%%%%%%%%%%%%%%%%%%%%%%%%%%%%%

%----------------------------------------------------------------------------------------
%	PACKAGES AND OTHER DOCUMENT CONFIGURATIONS
%----------------------------------------------------------------------------------------

\usepackage{amsmath, amsfonts, amsthm} % Math packages
\usepackage{hyperref}
\usepackage{comment}
\usepackage{amsmath}
\usepackage{bbm}
\usepackage{algpseudocode, algorithm, algorithmicx}
\usepackage[toc,page]{appendix}
\usepackage[user,titleref]{zref}

\hypersetup{
   colorlinks=true,
   citecolor=red,
   linkcolor=cyan,
   filecolor=magenta,
   urlcolor=blue,
 }

\usepackage{url}
\usepackage{listings} % Code listings, with syntax highlighting

\usepackage[english]{babel} % English language hyphenation

\usepackage{graphicx} % Required for inserting images
\graphicspath{{Figures/}{./}} % Specifies where to look for included images (trailing slash required)

\usepackage{booktabs} % Required for better horizontal rules in tables

\numberwithin{equation}{section} % Number equations within sections (i.e. 1.1, 1.2, 2.1, 2.2 instead of 1, 2, 3, 4)
\numberwithin{figure}{section} % Number figures within sections (i.e. 1.1, 1.2, 2.1, 2.2 instead of 1, 2, 3, 4)
\numberwithin{table}{section} % Number tables within sections (i.e. 1.1, 1.2, 2.1, 2.2 instead of 1, 2, 3, 4)

\setlength\parindent{0pt} % Removes all indentation from paragraphs

\usepackage{enumitem} % Required for list customisation
\setlist{noitemsep} % No spacing between list items

%----------------------------------------------------------------------------------------
%	DOCUMENT MARGINS
%----------------------------------------------------------------------------------------

\usepackage{geometry} % Required for adjusting page dimensions and margins

\geometry{
	paper=a4paper, % Paper size, change to letterpaper for US letter size
	top=2.5cm, % Top margin
	bottom=3cm, % Bottom margin
	left=3cm, % Left margin
	right=3cm, % Right margin
	headheight=0.75cm, % Header height
	footskip=1.5cm, % Space from the bottom margin to the baseline of the footer
	headsep=0.75cm, % Space from the top margin to the baseline of the header
	%showframe, % Uncomment to show how the type block is set on the page
}

%----------------------------------------------------------------------------------------
%	FONTS
%----------------------------------------------------------------------------------------

\usepackage[utf8]{inputenc} % Required for inputting international characters
\usepackage[T1]{fontenc} % Use 8-bit encoding

% \usepackage{fourier} % Use the Adobe Utopia font for the document

%----------------------------------------------------------------------------------------
%	SECTION TITLES
%----------------------------------------------------------------------------------------

\usepackage{sectsty} % Allows customising section commands

\sectionfont{\vspace{6pt}\centering\normalfont\scshape} % \section{} styling
\subsectionfont{\normalfont\bfseries} % \subsection{} styling
\subsubsectionfont{\normalfont\bfseries} % \subsubsection{} styling
\paragraphfont{\normalfont\scshape} % \paragraph{} styling

%----------------------------------------------------------------------------------------
%	HEADERS AND FOOTERS
%----------------------------------------------------------------------------------------

\usepackage{scrlayer-scrpage} % Required for customising headers and footers

\ohead*{} % Right header
\ihead*{} % Left header
\chead*{} % Centre header

\ofoot*{} % Right footer
\ifoot*{} % Left footer
\cfoot*{\pagemark} % Centre footer

%-----------------------------------------------------------
%  Define new commands
%-----------------------------------------------------------
\DeclareMathOperator*{\argmax}{arg\,max}
\DeclareMathOperator*{\argmin}{arg\,min}
\DeclareMathOperator\supp{supp}

\newcommand*\Let[2]{\State #1 $\gets$ #2}

\DeclareMathOperator*{\E}{\mathbbm{E}}
\DeclareMathOperator*{\Z}{\mathbbm{Z}}
\DeclareMathOperator*{\R}{\mathbbm{R}} % Include the file specifying the document structure and custom commands

%----------------------------------------------------------------------------------------
%	TITLE SECTION
%----------------------------------------------------------------------------------------

\title{
	\normalfont\normalsize
	\textsc{Harvard Privacy Tools Project}\\ % Your university, school and/or department name(s)
	\vspace{25pt} % Whitespace
	\rule{\linewidth}{0.5pt}\\ % Thin top horizontal rule
	\vspace{20pt} % Whitespace
	{\huge Geometric Mechanism Accuracy}\\ % The assignment title
	\vspace{12pt} % Whitespace
	\rule{\linewidth}{2pt}\\ % Thick bottom horizontal rule
	\vspace{12pt} % Whitespace
}

\date{\normalsize\today} % Today's date (\today) or a custom date

\begin{document}

\maketitle

\begin{definition}
Let $z$ be the true value of the statistic and let $X$ be the random variable the noisy release is drawn from. Let $\alpha$ be the statistical significance level, and let $Y = \vert X-z \vert.$ Then, accuracy $a$ for a given $\alpha \in [0,1]$ is the $a$ s.t.
$$ \alpha = \pr[Y > a].$$
\end{definition}

\begin{theorem}
The accuracy of an $\epsilon$-differentially private release from the geometric mechanism on a function with sensitivity $\Delta_1$, at statistical significance level $\alpha$ is
$$ a = \lceil \frac{\Delta_1}{\epsilon}\ln(1/\alpha) \rceil.$$
\end{theorem}\cite{balcer2017differential, ghosh2012universally}

\begin{proof}
This follows directly from the proof of accuracy for the Laplace mechanism, with the observation that the geometric mechanism is simply a discretization of the Laplace mechanism, hence the ceiling in the accuracy statement.
\end{proof}

\subsection{A note on converting from accuracy to privacy}

We offer the ability to convert from an accuracy guarantee to a privacy guarantee in our system. In the context of the geometric mechanism, it is not entirely clear what that conversion would mean, since for a set accuracy level $a$ and significance level $\alpha$, there are a range of possible values for $\epsilon$. Since this range of $\epsilon$ depends on both $a$ \textit{and} $\alpha$, rather than choosing the minimum or maximum over the range of $\epsilon$'s we instead use the original accuracy guarantee from the Laplace mechanism (i.e. the geometric mechanism's accuracy guarantee without the ceiling) to convert from statements about accuracy to statements about privacy:

$$ \epsilon = \frac{\Delta_1}{a}\ln(1/\alpha).$$

One might argue that in general we shouldn't take the ceiling when determining the accuracy, since the additional information might be useful to an end-user. For example, if a user is attempting to determine how much budget to give a query, and they first ask for accuracy for an $\epsilon$ that gives (without the ceiling) $a = 3.02$, they might determine to increase the amount of accuracy to get the accuracy guarantee to $a = 3$. If instead we take the ceiling, they would get back $a = 4$ and not know that a (theoretical) small change in $\epsilon$ would lead to a noticeable improvement in their accuracy guarantee. 

\bibliographystyle{alpha}
\bibliography{accuracies}
\end{document}