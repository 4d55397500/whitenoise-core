\documentclass[11pt]{scrartcl} % Font size
%%%%%%%%%%%%%%%%%%%%%%%%%%%%%%%%%%%%%%%%%
% Wenneker Assignment
% Structure Specification File
% Version 2.0 (12/1/2019)
%
% This template originates from:
% http://www.LaTeXTemplates.com
%
% Authors:
% Vel (vel@LaTeXTemplates.com)
% Frits Wenneker
%
% License:
% CC BY-NC-SA 3.0 (http://creativecommons.org/licenses/by-nc-sa/3.0/)
%
%%%%%%%%%%%%%%%%%%%%%%%%%%%%%%%%%%%%%%%%%

%----------------------------------------------------------------------------------------
%	PACKAGES AND OTHER DOCUMENT CONFIGURATIONS
%----------------------------------------------------------------------------------------

\usepackage{amsmath, amsfonts, amsthm} % Math packages
\usepackage{hyperref}
\usepackage{comment}
\usepackage{amsmath}
\usepackage{bbm}
\usepackage{algpseudocode, algorithm, algorithmicx}
\usepackage[toc,page]{appendix}
\usepackage[user,titleref]{zref}

\hypersetup{
   colorlinks=true,
   citecolor=red,
   linkcolor=cyan,
   filecolor=magenta,
   urlcolor=blue,
 }

\usepackage{url}
\usepackage{listings} % Code listings, with syntax highlighting

\usepackage[english]{babel} % English language hyphenation

\usepackage{graphicx} % Required for inserting images
\graphicspath{{Figures/}{./}} % Specifies where to look for included images (trailing slash required)

\usepackage{booktabs} % Required for better horizontal rules in tables

\numberwithin{equation}{section} % Number equations within sections (i.e. 1.1, 1.2, 2.1, 2.2 instead of 1, 2, 3, 4)
\numberwithin{figure}{section} % Number figures within sections (i.e. 1.1, 1.2, 2.1, 2.2 instead of 1, 2, 3, 4)
\numberwithin{table}{section} % Number tables within sections (i.e. 1.1, 1.2, 2.1, 2.2 instead of 1, 2, 3, 4)

\setlength\parindent{0pt} % Removes all indentation from paragraphs

\usepackage{enumitem} % Required for list customisation
\setlist{noitemsep} % No spacing between list items

%----------------------------------------------------------------------------------------
%	DOCUMENT MARGINS
%----------------------------------------------------------------------------------------

\usepackage{geometry} % Required for adjusting page dimensions and margins

\geometry{
	paper=a4paper, % Paper size, change to letterpaper for US letter size
	top=2.5cm, % Top margin
	bottom=3cm, % Bottom margin
	left=3cm, % Left margin
	right=3cm, % Right margin
	headheight=0.75cm, % Header height
	footskip=1.5cm, % Space from the bottom margin to the baseline of the footer
	headsep=0.75cm, % Space from the top margin to the baseline of the header
	%showframe, % Uncomment to show how the type block is set on the page
}

%----------------------------------------------------------------------------------------
%	FONTS
%----------------------------------------------------------------------------------------

\usepackage[utf8]{inputenc} % Required for inputting international characters
\usepackage[T1]{fontenc} % Use 8-bit encoding

% \usepackage{fourier} % Use the Adobe Utopia font for the document

%----------------------------------------------------------------------------------------
%	SECTION TITLES
%----------------------------------------------------------------------------------------

\usepackage{sectsty} % Allows customising section commands

\sectionfont{\vspace{6pt}\centering\normalfont\scshape} % \section{} styling
\subsectionfont{\normalfont\bfseries} % \subsection{} styling
\subsubsectionfont{\normalfont\bfseries} % \subsubsection{} styling
\paragraphfont{\normalfont\scshape} % \paragraph{} styling

%----------------------------------------------------------------------------------------
%	HEADERS AND FOOTERS
%----------------------------------------------------------------------------------------

\usepackage{scrlayer-scrpage} % Required for customising headers and footers

\ohead*{} % Right header
\ihead*{} % Left header
\chead*{} % Centre header

\ofoot*{} % Right footer
\ifoot*{} % Left footer
\cfoot*{\pagemark} % Centre footer

%-----------------------------------------------------------
%  Define new commands
%-----------------------------------------------------------
\DeclareMathOperator*{\argmax}{arg\,max}
\DeclareMathOperator*{\argmin}{arg\,min}
\DeclareMathOperator\supp{supp}

\newcommand*\Let[2]{\State #1 $\gets$ #2}

\DeclareMathOperator*{\E}{\mathbbm{E}}
\DeclareMathOperator*{\Z}{\mathbbm{Z}}
\DeclareMathOperator*{\R}{\mathbbm{R}} % Include the file specifying the document structure and custom commands

%----------------------------------------------------------------------------------------
%	TITLE SECTION
%----------------------------------------------------------------------------------------

\title{
	\normalfont\normalsize
	\textsc{Harvard Privacy Tools Project}\\ % Your university, school and/or department name(s)
	\vspace{25pt} % Whitespace
	\rule{\linewidth}{0.5pt}\\ % Thin top horizontal rule
	\vspace{20pt} % Whitespace
	{\huge Randomness and Noise}\\ % The assignment title
	\vspace{12pt} % Whitespace
	\rule{\linewidth}{2pt}\\ % Thick bottom horizontal rule
	\vspace{12pt} % Whitespace
}

\author{\LARGE Christian Covington} % Your name

\date{\normalsize\today} % Today's date (\today) or a custom date

\begin{document}

\maketitle

\section{Overview}
This document describes the strategies the library uses for generation of randomness and noise.
I believe there will need to be ongoing discussions about how best to perform randomized computations
in the library, as properly doing so is more complicated in practice than in theory.

% TODO: talk about exact rounding as post-processing of true DP

% TODO: note high variance in timing of mpfr (https://cseweb.ucsd.edu/~mandrysc/pub/subnormal.pdf)

% TODO: note that "precision" in mpfr is number of bits used in the mantissa (including the implicit leading 1)
%       and that calculations are run to ensure exact rounding regardless of the precision of the input/output

\section{Source of Randomness}
% TODO: make notes here on different models of noise generation?
All of our random number generation involves uniform random sampling of bits via OpenSSL.
We will take as given that OpenSSL is cryptographically secure.

\section{Preliminaries}
\begin{definition}
	\label{defn:differential_privacy}
	Differential Privacy \cite{DMNS06} \newline
	For $\epsilon, \delta \geq 0$, a randomized mechanism
	$\mathcal{M}: \mathcal{X}^n \times \mathcal{Q} \rightarrow \mathcal{Y}$ is
	$(\epsilon, \delta)$-DP if, for every pair of neighboring data sets $X, X' \in \mathcal{X}^n$ and
	every query $q \in \mathcal{Q}$ we have
	\[ \forall \mathcal{T} \subseteq \mathcal{Y}: \Pr[\mathcal{M}(X, \epsilon, \delta, q) \in \mathcal{T}] \leq e^{\epsilon} \Pr[\mathcal{M}(X', \epsilon, \delta, q) \in \mathcal{T}] + \delta. \]
\end{definition}
If $\delta = 0$, we call this \emph{Pure DP}. If $\delta > 0$, we call this \emph{Approximate DP}.
Note that, in practice, differential privacy could be thought of a bit more broadly -- as a bounded
distance between joint distributions over the mechanism output and runtime.\footnote{Conceivably, this idea
could be extended further to talk about distributions over all quantities related in any way to the underlying data.
For example, imagine that the US government uses $\epsilon = 1$ if the President is in the data and $\epsilon = 10$
if not -- if anyone knew about this rule, the choice of epsilon would leak information not accounted for in the
traditional definition of DP.
We will focus only on mechanism output and runtime, as they seem to be by far the most plausible
channels of information leakage.} We will focus mostly on the distribution over mechanism output, as this
is really the core idea of DP, but will touch on runtime when it seems appropriate.

% \begin{theorem}
% 	\label{thm:post_processing}
% 	Post-Processing \newline
% 	Let $f: \mathcal{Y} \rightarrow \mathcal{Y}'$ be an arbitrary mapping independent
% 	of the data set $X$ and
% 	$\mathcal{M}$ be an $(\epsilon, \delta)$-DP mechanism.
% 	Then $f \circ \mathcal{M}$ is also $(\epsilon, \delta)$-DP.

% 	\begin{proof}
% 		This is a well-known result -- see Proposition 2.1 on pg. 19 of \cite{DR14} for a proof.
% 	\end{proof}
% \end{theorem}

\begin{definition}
	\label{defn:exact_rounding}
	Exact Rounding \newline
	Let $S \subset \R$ be some set.
	Let $\phi: \R^n \rightarrow \R$ be a function on the reals and $\phi': S^n \rightarrow S$ be its implementation over $S$.
	Then, $\phi'$ respects \emph{exact rounding} for $(\phi, S)$ if
	\[ \forall s \in S: \phi'(s) = round_{S}[\phi(s)], \]
	where $round_{S}(\cdot)$ rounds a real number to a member of $S$ according to some rounding rule.
\end{definition}
For our purposes, we will care only about the case where $S = \F$, the set of IEEE-754 floating-point numbers.
% We may occasionally use a different definition of $S$ for ease of reasoning about a proof, or refer to $\F$
% specifically to more clearly contextualize what we are trying to do.
% Additionally, our rounding rule will typically be that we want to round our real number $r$ to the $s \in S$ that minimizes
% $\vert r-s \vert$.

% \begin{corollary}
% 	\label{cor:exact_rounding_post_processing}
% 	Exact Rounding as Post-Processing \newline
% 	Let $\mathcal{M}: \R^n \times \mathcal{Q} \rightarrow \R$ be
% 	$(\epsilon, \delta)$-DP and $\mathcal{M}': S^n \times \mathcal{Q} \rightarrow S$
% 	be an implementation over $S$. If $\mathcal{M}'$ respects exact rounding for
% 	$(\mathcal{M}, S)$, then $\mathcal{M}'$ is $(\epsilon, \delta)$-DP.

% 	\begin{proof}
% 		$\mathcal{M}'$ can be viewed as a function that takes real-valued outputs
% 		from $\mathcal{M}$ and rounds them to an element of $S$ according to a rounding rule.
% 		This is post-processing independent of the data $X$, and so the corollary follows
% 		directly from Theorem~\ref{thm:post_processing}.
% 	\end{proof}
% \end{corollary}

% \begin{theorem}
% 	\label{thm:non_composability_exact_rounding}
% 	Non-composability of Exact Rounding \newline
% 	Let $\phi, \eta$ be functions on $\R$ and $\phi', \eta'$ be versions of $\phi,\eta$
% 	that respect exact rounding with respect to some $S \subset \R$.
% 	Then, $\phi' \circ \eta'$ does not necessarily respect exact rounding with respect to $S$.

% 	\begin{proof}
% 		For ease of proof, we assume WLOG that $S = \Z$. Now, let $\phi: \R \times \R \rightarrow \R$ be such that
% 		$\phi(a,b) = a+b$ and $\eta: \R \times \R \rightarrow \R$ be such that $\eta(a,b) = a/b$.
% 		Let's examine the behavior of $\phi \circ \eta$ on a certain set of inputs:
% 		\begin{align*}
% 			\phi \left( \eta(3, 2), \eta(6, 5) \right) &= \phi \left( 3/2, 6/5 \right) \\
% 													   &= 27/10.
% 		\end{align*}
% 		Note that, for these inputs, exact rounding of $\phi \circ \eta$ with respect to $\Z$ would yield an output of $3$. \newline

% 		We now consider $\phi' \circ \eta'$, where $\phi': \Z \times \Z \rightarrow \Z$ and
% 		$\eta': \Z \times \Z \rightarrow \Z$ each respects exact rounding with respect to $\Z$.
% 		\begin{align*}
% 			\phi' \left( \eta'(3, 2), \eta'(6, 5) \right) &= \phi' \left( 1, 1 \right) \\
% 														  &= 2.
% 		\end{align*}
% 		$2 \neq 3$, so we have our proof.
% 	\end{proof}
% \end{theorem}
% The non-composability of exact rounding means that translating DP algorithms from $\R$ to $\F$
% is not as easy as combining individual steps that respect exact rounding, as doing so
% does not imply that the mechanism as a whole respects exact rounding.
% In the next section, we will propose a weaker sufficient condition for DP on $\F$.

% \begin{definition}
% 	\label{def:additive_noise_mechanism}
% 	Additive Noise Mechanism \newline
% 	A differentially private mechanism $\mathcal{M}$ is an \emph{additive noise mechanism}
% 	if it is of the form
% 	\[ \mathcal{M}(X, \phi) = \phi(X) + n, \]
% 	where $n$ is a draw from a noise distribution $N$.
% \end{definition}

\begin{definition}
	\label{def:truncation_censoring}
	Truncation and Censoring \newline
	Throughout our noise functions, we use the terms ``truncated'' and ``censored''.
	Both are means of bounding the support of the noise distribution, but they are distinct. \newline

	Truncating a distribution simply ignores events outside of the given bounds, so
	all probabilities within the given bounds are scaled up by a constant factor.
	One way to generate a truncated distribution is via rejection sampling.
	You can generate samples from a probability distribution as you normally would (without any bounding),
	and reject any sample that falls outside of your bounds. \newline

	Censoring a distribution, rather than ignoring events outside of the given bounds, pushes the
	probabilities of said events to the closest event within the given bounds. One way to generate
	a censored distribution would be to generate samples from a probability distribution as you
	typically would, and then clamp samples that fall outside of your bounds to the closest element
	inside your bounds.
\end{definition}

\section{Current Random Number Generation}
We have a set of fairly standard procedures for generating draws from various noise distributions.

\subsection{Introduction to MPFR}
The \href{https://www.mpfr.org/}{GNU MPFR Library}\cite{FHL+07} is a C library with methods for carrying out a number
of floating-point operations with \emph{exact rounding}.
MPFR has methods for, among other things, performing basic arithmetic operations and generating samples
from basic noise distributions. \newline 

\subsubsection{Use in the library}
We use MPFR to sample from Uniform[0,1] and Gaussian(0,1) with exact rounding. We are currently thinking about what exact rounding can buy us in terms 
of ensuring differential privacy in practice, but for now are using these only as accurate sampling mechanisms. \newline 

% \subsubsection{$sample\_uniform(min: f64, max: f64)$}
% In this method, we start by generating a floating-point number in $[0,1)$,
% where each is generated with probability relative to its unit of least precision (ULP).\footnote{The ULP is the value
% represented by the least significant bit of the mantissa if that bit is a 1.}
% That is, we generate $x \in [2^{-i}, 2^{-i+1})$ with probability $\frac{1}{2^i}$
% for all $i \in \{1,2,\hdots,1022\}$ and $x \in [0, 2^{-1022})$ for $i = 1023$.

% Within each precision band (the set of numbers with the same unit
% of least precision), numbers are sampled uniformly.
% We achieve this sample our exponent from a geometric distribution with parameter $p = 0.5$ and a mantissa uniformly from $\{0,1\}^{52}$.
% Let $e$ be a draw from $Geom(0.5)$ (truncated such that $e \in \{1,2,\hdots,1023\}$) and $m_1, m_2 \hdots, m_{52}$ be the bits of our mantissa.
% At the end, we will scale our output from $[0,1)$ to be instead in $[min, max)$. Then our function outputs $u$, where
% \[ u = (1.m_1m_2 \hdots m_{52})_2 * 2^{-e} * (max - min) + min. \]

% This method was proposed in \cite{Mir12} as a component of a larger attempt to create
% a version of the Laplace mechanism that is not susceptible to floating-point attacks.\footnote{Note that the
% original method generates values $\in [0,1)$ rather than arbitrary $[min, max)$.}
% There is no universally agreed upon method for generating uniform random numbers (for privacy
% applications or otherwise), but this method seems to approximate the real numbers better than many
% other common methods because of the sampling relative to the ULP. \newline

% \begin{tcolorbox}[colback = {green}, title = {Known Privacy Issues}, colbacktitle = black]
% 	When $i=1023$ we are sampling from subnormal floating-point numbers. Because processors do not typically support
% 	subnormals natively, they take much longer to sample and open us up to an easier timing attack, as
% 	seen in \cite{AKM+15}.
% 	Protecting against timing attacks is mostly seen as out of scope for now, but I wanted
% 	to bring this up anyway. \newline

% 	We are incurring some floating-point error when converting from $[0,1)$ to $[min, max)$ which
% 	could jeopardize privacy guarantees in ways that are difficult to reason about.\cite{Mir12} \cite{Ilv19}
% \end{tcolorbox}

\subsection{Biased Bit Sampling}
Recall that we are taking as given that we are able to sample uniform bits from OpenSSL.
For many applications, however, we want to be able to sample bits non-uniformly,
i.e. where $\Pr(bit = 1) \neq \frac{1}{2}$. To do so, we use the $sample\_bit$ function.

\subsubsection{$sample\_bit(prob: f64)$}
This function uses the unbiased bit generation from OpenSSL to return a single bit, where $\Pr(bit = 1) = prob$.
I was introduced to the method for biasing an unbiased coin from a homework assignment given by Michael Mitzenmacher,
and I later found a write-up online \href{https://amakelov.wordpress.com/2013/10/10/arbitrarily-biasing-a-coin-in-2-expected-tosses/}{here}.
We will give a general form of the algorithm, and then talk about implementation details.
\begin{algorithm}[H]
	\caption{Biasing an unbiased coin}
	\label{alg:biasing_a_coin}
	\begin{algorithmic}[1]
		\State $p \gets \Pr(bit = 1)$
		\State Find the infinite binary expansion of $p$, which we call $b = (b_1, b_2, \hdots,)_2$.
		Note that $p = \sum_{i=1}^{\infty}\frac{b_i}{2^i}$.
		\State Toss an unbiased coin until the first instance of ``heads''. Call the (1-based) index where this occurred $k$.
		\State return $b_k$
	\end{algorithmic}
\end{algorithm}
Let's first show that this procedure gives the correct expectation:
\begin{align*}
	p &= \Pr(bit = 1) \\
		 &= \sum_{i=1}^{\infty} \Pr(bit = 1 \vert k = i) \Pr(k = i) \\
		 &= \sum_{i=1}^{\infty} b_i \cdot \frac{1}{2^i} \\
		 &= \sum_{i=1}^{\infty}\frac{b_i}{2^i}.
\end{align*}
This is consistent with the statement in Algorithm~\ref{alg:biasing_a_coin}, so we know that
the process returns bits with the correct bias.
In terms of efficiency, we know that we can stop coin flipping once we get a heads,
so that part of the algorithm has $\E(\# flips) = 2$. \newline

The part that is a bit more difficult is constructing the infinite binary expansion of $p$.
We start by noting that, for our purposes, we do not actually need an infinite binary expansion.
Because $p$ will always be a 64-bit floating-point number, we need only get a binary
expansion that covers all representable numbers in our floating-point standard that are
also valid probabilities.
Luckily, the underlying structure of floating-point numbers makes this quite easy. \newline

In the 64-bit standard, floating-point numbers are represented as
\[ (-1)^s(1.m_1m_2 \hdots m_{52})_2 * 2^{(e_{1}e_2 \hdots e_{11})_2 - 1023}, \]
where $s$ is a sign bit we ignore for our purposes.
Our binary expansion is just the mantissa $(1.m_1m_2 \hdots m_{52})_2$, with
the radix point shifted based on the value of the exponent.
We can then index into the properly shifted mantissa and check the value of the $k$th element.

\subsection{Other Continuous Distributions}
In general, we can generate draws from non-uniform continuous distributions (e.g. Laplace)
by using \href{https://en.wikipedia.org/wiki/Inverse_transform_sampling}{inverse transform sampling}.
To draw from a distribution $f$ with CDF $F$,
we sample $u$ from $Unif[0,1)$ and return $F^{-1}(u)$. \newline

\begin{tcolorbox}[colback = {green}, title = {Known Privacy Issues}, colbacktitle = black]
	Carrying out the inverse probability transform employs floating-point arithmetic,
	so we run into the same problems as were described in the uniform sampling section.
	This is potentially a very significant problem, and one for which we do not
	currently have a good solution.
\end{tcolorbox}
Because of the vulnerabilities inherent in using floating-point arithmetic, we would
like to avoid using inverse transform sampling when possible.

\subsection{Geometric Distribution}
The Geometric is one such case where we can generate a distribution without inverse transform sampling.
To generate a $Geom(p)$, we can use our $sample\_bit$ function to
repeatedly sample random bits where $\Pr(bit = 1) = p$. We then return the number of samples
it takes to get our first $1$. This method is not susceptible to attacks based on floating-point vulnerabilities,
as it operates only over the integers. \newline

% \section{Sufficient Conditions for DP in Practice}
% Given the existence of the aforementioned floating-point vulnerabilities in standard mechanism implementations,
% we would like to explore ways to potentially ensure that implementations actually respect DP in practice.
% This section is very experimental/preliminary.



% \subsection{Additive Noise Mechanisms}
% Let $\phi$ be a function we would like to privatize at the $(\epsilon, \delta)$ level,
% with sensitivity $\Delta_{\phi}$.\footnote{This is the maximum the output of
% the function can differ, in terms of some distance metric, when evaluated on neighboring data sets.} \newline

% Let's start by considering the generation of noise according to some distribution $N_{\epsilon, \delta, \Delta_{\phi}}$,
% which for ease of notation we will call $N$.
% For our theoretical proofs of many common mechanisms we assume that $supp(N) = \R$,
% but in practice $supp(N) = \F$. It is well-established that privacy properties
% of noise distributions defined on $\R$ do not necessarily hold on $\F$.
% See \cite{Mir12} and \cite{Ilv19} for vulnerabilities of the Laplace and Exponential
% mechanisms, respectively, and \cite{GMP16} for a more general treatment. \newline

% \begin{theorem}
% 	\label{thm:valid_noise_on_R}
% 	Valid Noise for Additive Noise Mechanisms on $\R$ \newline
% 	A noise distribution $N$ with $supp(N) = \R$ can be used as the noising portion of an additive noise mechanism to privatize $\phi$ if
% 	$\forall \mathcal{T} \in supp(N)$
% 	\[ \frac{\Pr(N \in \mathcal{T})}{\Pr(N \in \mathcal{T} - \Delta_{\phi})} \leq \ e^{\epsilon} + \delta \]
% 	and
% 	\[ \frac{\Pr(N \in \mathcal{T})}{\Pr(N \in \mathcal{T} + \Delta_{\phi})} \leq \ e^{\epsilon} + \delta, \]
% 	where $\mathcal{T} \pm \Delta_{\phi} = \{t \pm \Delta_{\phi} | t \in \mathcal{T}\}$.

% 	\begin{proof}
% 		This follows directly from Theorem~\ref{defn:differential_privacy}.
% 		$\mathcal{M}(X, \epsilon, \delta, q) = \phi(X) + N$ (and likewise for $X'$),
% 		and we can just ignore the $\phi(X), \phi(X')$ terms because the relationship between
% 		$\mathcal{M}(X, \epsilon, \delta, q)$ and $\mathcal{M}(X', \epsilon, \delta, q)$
% 		does not change if you shift both by the same factor.
% 		The $\pm \Delta_{\phi}$ terms are standing in for the notion of neighboring data sets.
% 	\end{proof}
% \end{theorem}

% \begin{corollary}
% 	\label{cor:valid_noise_on_F}
% 	Valid Noise for Additive Noise Mechanisms on $\F$ \newline
% 	Let $N$ be a valid noise distribution with $supp(N) = \R$ as described in Theorem~\ref{thm:valid_noise_on_R}
% 	and $N'$ be an implementation of $N$ that respects exact rounding with respect to $\F$.
% 	Then $N'$ is valid to be used in an additive noise mechanism on $\F$.

% 	\begin{proof}
% 		First, notice that in Theorem~\ref{thm:valid_noise_on_R}, we have
% 		effectively made the claim that if some property holds for an additive noise mechanism,
% 		it holds for the distribution of noise.
% 		Now, our result follows from Corollary~\ref{cor:exact_rounding_post_processing}.
% 	\end{proof}
% \end{corollary}

% \begin{theorem}
% 	\label{thm:additive_noise_on_F}
% 	Additive Noise Mechanisms on $\F$ \newline
% 	Let $\mathcal{M}(X, \epsilon, \delta, q) = \phi(X) + N$ be an $(\epsilon, \delta)$-DP additive noise mechanism
% 	defined on $\R$.
% 	Let $\phi'$ be the floating-point implementation of $\phi$,
% 	$N'$ be a valid (in the sense of Corollary~\ref{cor:valid_noise_on_F})
% 	implementation of $N$ on $\F$, and $\mathcal{M'}(X, \epsilon, \delta, q) = \phi'(X) + N'$
% 	respect exact rounding for $(\phi'(X)+N', \F)$.
% 	Then, $\mathcal{M'}(X, \epsilon, \delta, q)$ respects $(\epsilon, \delta)$-DP.

% 	\begin{proof}
% 		We know that $N'$ is a valid noise distribution, i.e. that it has the property (bounded distance at
% 		intervals of $\Delta_{\phi}$) we need for an additive noise mechanism.
% 		So if we imagine an idealized notion of a function $\phi'(X) + N'$ with outputs in $\R$,
% 		this function would be $(\epsilon, \delta)$-DP because we know that $\phi(X) + N$ is $(\epsilon, \delta)$-DP.
% 		Our mechanism $\mathcal{M'}(\cdot)$ respects exact rounding for $(\phi'(X) + N', \F)$, and so
% 		$\mathcal{M'}(\cdot)$ is $(\epsilon,\delta)$-DP by Corollary~\ref{cor:exact_rounding_post_processing}.
% 	\end{proof}
% \end{theorem}
% The upshot of Theorem~\ref{thm:additive_noise_on_F} is that the floating-point implementation of
% an additive noise mechanism is DP if both the noise generation and addition steps can be performed
% with exact rounding relative to $\F$.

% % \subsection{Use in the library}
% %TODO: do this section

\bibliographystyle{alpha}
\bibliography{noise}

\end{document}